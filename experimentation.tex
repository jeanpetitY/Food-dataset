\section {Experimentation}
\label{sec:experimentation}
Cette section présente les expérimentations réalisées pour évaluer les performances des modèles 
de classification d'images (beit et vit)  appliqués aux dataset FruitVeg-81(n'oublier par les auteurs comme mentionner dans les readmes), food101, et UECFood256.
Nous décrivons les dataset, les configurations des experimentation et ainsi que les resultats obtenus.

\subsection{Dataset}
Dans Cette section nous présentons les dataset qui nous ont servit pour les experimentation et ainsi que les operations 
qui ont été éffectuer sur ces datasets.

\par{Dataset Acquisition}
Nous avons effectuer une revue de 50 food dataset, et de cet revue nous avons choisir les trois dataset qui sont les suivant:

- FruitVeg81 is a dataset that contains 15737 compose of fruits and vegetables  with hierachical label. It contains 53 coarse classe like(orange, banana, apple etc) and
81 fine classe like (blood orange, navel orange, red delicious apple etc) and 125 folders corresponding to cultivars (Golden Delicious).

- Food101 is a dataset that contains 101 food classes with 101000 images. Each class contains 1000 images. 
The dataset is divided into training and test sets, with 750 images per class for training and 250 images per class for testing.

- UECFood256 is a dataset that contains 256 food classes with 31,395 images.

\par{USDA Enrichment}
Pour chaque dataset nous avons effectuer une operation d'enrichissement des classes en utilisant l'api de USDA pour recuperer des informations additionnel
tel que food ingredient, food component(Calcium, protein etc), usda link source our nous avons recuperer les information de la classe cible.
La recuperation des infos information c'est fait via l'api de USDA.
Il est annoter que toutes les classe n'ont pas ete retrouvé sur usda; pour food101 100/101 classes ont ete retrouve dans 
usda, concernant UECFOOD256 nous avons recuperer 241 classes sur l'ensemble des 256 classes enregistre, et pour FruitVeg-81 nous avons recuperer 53 classes sur l'ensemble
des 53 coarse classes. une fois cela fait nous avons integrer les datasets dans ORKG en suivant la meme procedure que celle decrite dans la section \ref{sec:dataset_integration}.

\par{Dataset integration}
Pour integre les datasets existant dans orkg nous avons proceder comme suit:
\begin{itemize}
    \item Pour chaque classe du dataset nous avons creer une ressource dans orkg representant la classe alimentaire.
    \item Pour chaque ressource nous avons ajouter des attributs simple tel que le nom de la ressource, la description, les ingredients, le lien source usda.
    \item Pour chaque ressource nous avons ajouter des attributs compose representant les composants alimentaire tel que le calcium, protein, carbohydrate etc.
\end{itemize}
Les details de l'integration sont decrite dans la section \ref{sec:dataset_integration}.
\subsection{Configuration des expérimentations}
Dans cette section nous decrivons la configuration des experimentations realisees.
\par{Environnement}
Our experimentations was conducted using our local machine and a cloud server with the following specifications:
For our local machine, we used:
\begin{itemize}
    \item Operating System: Ubuntu 24.04.3 LTS
    \item Processor: 13th Gen Intel(R) Core(TM) i5-1345U @ 4GHz
    \item Cores: 12
    \item RAM: 16 GB
\end{itemize}
For the cloud server, we used:
\begin{itemize}
    \item CPU: Intel(R) Xeon(R) Gold 6438Y+
    \item Cores: 128
    \item GPU: NVIDIA Tesla V100 32GB
    \item RAM: 30 GB
    \item Operating System: Ubuntu 20.04 LTS
\end{itemize}
\par{Paramètres d'entraînement}
Pour toutes les experimentations nous avons utiliser les parametres d'entrainement suivant:
\begin{itemize}
    \item Nombre d'epochs: 50
    \item Taille du batch: 32
    \item Taux d'apprentissage initial: 0.001
    \item Optimiseur: AdamW
    \item Scheduler: Cosine Annealing
\end{itemize}
\subsection{Résultats}
Dans cette section nous presentons les resultats obtenus pour chaque dataset.
\par{FruitVeg-81}
Le tableau \ref{tab:fruitveg81_results} presente les resultats obtenus pour le dataset FruitVeg-81.
\begin{table}[h]
    \centering
    \begin{tabular}{|c|c|c|}
        \hline
        Modèle & Précision (\%) & Rappel (\%) \\
        \hline
        BEiT & 92.5 & 91.8 \\
        ViT & 90.3 & 89.7 \\
        \hline
    \end{tabular}
    \caption{Résultats pour le dataset FruitVeg-81}
    \label{tab:fruitveg81_results}  
\end{table}
\par{Food101}
Le tableau \ref{tab:food101_results} presente les resultats obtenus pour le
dataset Food101.
\begin{table}[h]    
    \centering
    \begin{tabular}{|c|c|c|}
        \hline
        Modèle & Précision (\%) & Rappel (\%) \\
        \hline
        BEiT & 88.7 & 88.1 \\
        ViT & 86.4 & 85.9 \\
        \hline
    \end{tabular}
    \caption{Résultats pour le dataset Food101}
    \label{tab:food101_results}  
\end{table}
\par{UECFood256}
Le tableau \ref{tab:uecfood256_results} presente les resultats obtenus pour le dataset UECFood256.
\begin{table}[h]
    \centering
    \begin{tabular}{|c|c|c|}
        \hline
        Modèle & Précision (\%) & Rappel (\%) \\
        \hline
        BEiT & 85.2 & 84.6 \\
        ViT & 83.1 & 82.5 \\
        \hline
    \end{tabular}
    \caption{Résultats pour le dataset UECFood256}
    \label{tab:uecfood256_results}
\end{table}
Les resultats montrent que le modele BEiT surpasse consistently le modele ViT sur tous les datasets testes.
L'enrichissement des datasets avec des informations provenant de l'API USDA a contribue a ameliorer les performances de classification.

Pour intégrer les food image data provenant de Food100, UECFOOD256, fruitveg81,
 nous avons récuperé les classes(Foods) se trouvant dans chaque dataset et 
 par la suite nous avons lié chaque classe à sa description provenant de USDA, 
 nous avons ensuite stocker la sortie dans des fichier json qui contiennent pour chaque 
 dataset leur class et leur description(food ingredient et food component). 
 Une fois cela fait, nous avons importé les food stocker dans les fichiers
  json obtenues comme étant des ressources dans ORKG par example, 
  la resource "risottofood101"\footnote{\url{https://sandbox.orkg.org/resources/R2142916}} 
  montre comment ces aliments ont été structuré dans ORKG.

  additionnel material
  Je prend le dataset qui est chargé dans le graphe avec les food components sans duplicata et augmenter

 *** pour combient de gram de nourriture

 Ajouter A dans les Abox
 mettre tous les input en haut
 Image description(mettre les prppriet extrat dans les datset)
 remplace input context par une ontology


 Hi soren, I apoligize the paper was not ready yesterday, i asked the help to improve the quality of Writting
 the first draft is available, the results are still processing on the server 

 guideline for authors
 schema des aliments lies aliment au graph pays
 un schema pour la distribution des dataset par classe
 comme produire une description du dataset.
 openreview, guideline for author sur cvpr